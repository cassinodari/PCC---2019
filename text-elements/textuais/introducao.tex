% INTRODUÇÃO------------------------------------------------------------------

\chapter{INTRODUÇÃO}
\label{chap:introducao}


Atualmente a tecnologia está fazendo parte do dia a dia das pessoas e o maior desafio é sabermos aproveitar esta ferramenta que nos proporciona muitas coisas boas. A tecnologia tem movimentado bastante a área da agricultura onde vem sendo feito muitos negócios inovadores. Os produtores também são muito favorecidos em poder utilizar a tecnologia à seu favor, podendo assim crescer cada dia mais. 
No período atual da sociedade, o acesso a informação é algo indispensável, e com o agronegócio isso também acontece, quanto mais informação, maior crescimento a fazenda, empresa ou negócio pode ter, tornando-se mais valiosa. 

O desenvolvimento da aplicação proposta, consiste em auxiliar o pequeno e médio produtor rural a fazer o controle e manter o histórico sobre um determinado lote de frangos. Algumas informações serão coletadas uma única vez no início do lote que são: data e hora de recebimento das aves, quantidade alojadas e mortas no descarregamento. E outros dados serão coletados no dia a dia do produtor como por exemplo: mortalidade, recebimento de ração, peso médio dos frangos, entre outros.

Através de um aplicativo desenvolvido em React Native, o produtor poderá registrar todas essas informações de forma didática e com fácil usabilidade, podendo fazer a consulta e gerar relatórios desses dados em uma plataforma que estará disponível na web.
Uma vez que estas informações estarão em fácil acesso para o produtor, o controle torna-se mais fácil de ser realizado, já que para lançar esses dados teria que ser feito manualmente através de papéis.

A grande vantagem de poder registrar em um aplicativo móvel é que o produtor poderá ter a informação em mãos sempre que precisar, não necessitando ter todos os papéis no momento em que precisar fazer uma consulta de rápido acesso.

Na sequência serão apresentados os principais objetivos do trabalho, a justificativa, a base do trabalho está descrito no referencial teórico, seguido da metodologia utilizada para atingir os objetivos propostos e também a representação do cronograma.