% INTRODUÇÃO------------------------------------------------------------------

\chapter{INTRODUÇÃO}
\label{chap:introducao}




Atualmente a tecnologia está fazendo parte do dia a dia das pessoas e o maior desafio é saber aproveitar esta ferramenta que nos proporciona muitas coisas boas. A inovação no agronegócio está evoluindo constantemente e com isso o produtor rural é favorecido, pois através da tecnologia aplicada às fazendas e empresas é possível ter o crescimento almejado, contando sempre com a tecnologia em seu favor.

O desenvolvimento da aplicação proposta consiste em auxiliar o pequeno e médio produtor rural a fazer o controle e manter o histórico sobre um determinado lote de frangos. Algumas informações serão coletadas uma única vez, no início do lote, são elas: data e hora de recebimento das aves, quantidade alojada e quantidade de frangos mortos no descarregamento. Alguns outros dados serão coletados no dia a dia do produtor, como por exemplo: mortalidade, recebimento de ração, peso médio dos frangos, entre outros.

Através de um aplicativo mobile que será desenvolvido em React Native, o produtor poderá registrar todas as informações de forma didática e com fácil usabilidade. Também será construído uma plataforma na web que servirá para fazer consultas e gerar relatórios dos dados registrados através do aplicativo.
Uma vez que estas informações estarão em fácil acesso para o produtor através do próprio aplicativo ou ainda da página disponível na web, o controle do investimento acaba tornando-se mais eficaz.


A grande vantagem de poder registrar em um aplicativo móvel é que o produtor poderá ter a informação em mãos sempre que precisar, não necessitando ter todos os papéis no momento em que precisar fazer uma consulta rápida.

Na sequência serão apresentados os principais objetivos do trabalho, a justificativa, a base do trabalho descrita no referencial teórico, seguido da metodologia utilizada para atingir os objetivos propostos, representação do cronograma, resultados esperados com o desenvolvimento do trabalho, além dos apêndices e também dos anexos.