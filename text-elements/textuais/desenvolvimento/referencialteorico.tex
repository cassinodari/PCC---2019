
\chapter{REFERENCIAL TEÓRICO}
\label{chap:ref_teo}

\section{Ferramentas de desenvolvimento}
\subsection{JSX}

É uma extensão de sintaxe para \textit{JavaScript}, recomendada para auxílio na construção de interface gráfica e que será utilizada com o \textit{React} e também com o \textit{React Native} para falicitar a estrutura de códigos durante o desenvolvimento do \textit{front-end}. O \textit{JSX} necessita de um transpilador e o mais utilizado é o \textit{Babel} e a sintaxe dele é muito semelhante à do \textit{XML}.  
\cite{jsx}.

\subsection{React Native}

Para o desenvolvimento móvel do trabalho proposto, será utilizado o framework \textit{React Native}, que é um projeto \textit{open-source} desenvolvido por engenheiros do Facebook. Com este framework é possível criar aplicativos que rodam no sistema operacional Android e ao mesmo tempo, em sistema operacional iOs. Esta característica é chamada híbrida, ou seja, quando é possível o funcionando em mais de uma plataforma. Com a tecnologia proposta serão utilizados componentes nativos para fazer a compilação de um \textit{APP}, tornando-o assim um aplicativo que fica no meio termo entre aplicativo nativo e híbrido, já que comporta componentes nativos mas também é flexível para executar o projeto em mais de uma plataforma.

A mão de obra acaba sendo mais cara para programadores que desenvolvem para uma única plataforma, e por este motivo diversas empresas pensam bastante antes de dar início a construção de algo que irá tomar mais tempo e terá um custo maior, já que terá que desenvolver dois projetos simultaneamente. Muitas dessas empresas acabam obtando por desenvolver uma única aplicação que possa abrangir as duas plataformas, e no caso deste trabalho, também foi escolhido o React Native para facilitar a construção do projeto em ambos sistemas operacionais. 

React Native é escrito na linguagem \textit{JavaScript}, que é a mais utilizada no mundo atualmente e segundo \citeonline{javascript}, é uma linguagem de comportamento que permite fazer a criação de conteúdos dinâmicos, entre outros. Consiste em ser uma linguagem de alto-nível, padronizada, é uma ferramenta para desenvolvimento de projetos complexos e também age com facilidade em testar o próprio código de forma muito rápida 

\subsection{Node.JS}
Sendo o \textit{NodeJS} uma plataforma para construir aplicações web escaláveis de alta performance usando JavaScript e trabalha ao lado do servidor, é que foi escolhido para o desenvolvimento da \textit{API}. \citeonline{node} ressalta que ele integra muito bem com javascript e pode receber muitas requisições sem que haja perca de desempenho, já que utilizando uma única thread ele pode dar conta do número de requisições. A ideia do Node.JS era ser apenas um servidor, mas hoje é possível montar servidores \textit{http} e \textit{https}, bem como \textit{DNS, TCP, Media Server} entre outros.

O líder de soluções \textit{open source} \citeonline{api}, diz que uma \textit{API} é uma interface de programação de aplicações, onde permite integrar \textit{API}'s de outros softwares, podendo assim, realizar a comunicação entre eles. Com este conceito dito, será realizado a integração do \textit{back-end} com a aplicação \textit{front-end} da web e será feita a comunicação entre elas, juntamente com a comunicação do aplicativo.

\subsection{NestJS}
O \textit{NestJS} foi escolhido dentre as tecnologias, para o desenvolvimento proposto, pois entende-se que pode diminuir o desenvolvimento de erros em tempo de execução, já que o servidor NestJS é compilado sobre um servidor Node.JS. Ele é uma estrutura utilizada para criar aplicativos ao lado do servidor Node.JS eficientes e escaláveis \citeonline{node1}. Esta ferramenta se diferencia na oferta de arquitetura de aplicativos pronta para a criação de aplicativos altamente testáveis, escalonáveis e de fácil manutenção.

\subsection{React}
\textit{React} se diferencia de \textit{React Native} por ser uma plataforma voltada para aplicações web. Também pode ser utilizado \textit{JSX} pra facilitar na estrutura da interface do usuário e é uma tecnologia nova que está crescendo muito, também foi criada pelos engenheiros do Facebook e é mantida por comunidades. A escolha da utilização do \textit{React} para codificar o painel web, se deu devido a facilidade de utilização, já que é um framework fracamente acoplado.




...