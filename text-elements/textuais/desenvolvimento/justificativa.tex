% JUSTIFICATIVA ------------------------------------------------------------------

\chapter{JUSTIFICATIVA}
\label{chap:justificativa}

Sabemos que na sociedade moderna as pessoas ainda possuem certa resistência com relação à tecnologia, optando assim pela utilização de recursos mais conhecidos. Isto acontece pois quanto mais antiga a geração, mais dificuldade com a tecnologia elas terão, os tornando mais resistentes aos benefícios que ela nos proporciona. Esta resistência também acontece em vários setores da indústria, sendo a agroindústria também afetada. 

Um fato bem importante que pode ser ressaltado é a necessidade de estar conectado. O Brasil está na quarta posição mundial do Top 5 no acesso \textit{mobile}. A empresa norte americana App Anie fez um levantamento sobre o uso de aplicativos.
O relatório afirma que em 2018 foram feitos 194 bilhões de downloads no mundo e gastos 101 bilhões de dólares em lojas de aplicativos. Além disso, o tempo médio gasto em apps aumentou 50\% de 2016 para 2018, chegando, a 3 horas por dia em média. Esses dados são muito significantes e nos diz que cada vez mais as pessoas estão fazendo o uso da ferramenta móvel, falicitando o acesso à informação.

Poder trazer para a visão do produtor de aves de corte maior agilidade e praticipadade, proporcionando bons resultados no momento do período de chegada, alojamento e carregamento das aves de corte, juntamente com o controle diário, é que cria-se uma expectativa muito grande para o desenvolvimento deste projeto.

O problema tratado neste trabalho é o fato de que tarefas como o registro dos dados coletados diariamente ainda são realizados no papel, tornando os dados dos lotes mais desorganizados e tendo um histórico mais difícil de ser acessado. Além disso o produtor não consegue ter as informações consigo, já que isso fica armazenado em folhas de papel.

A importância de possuir estes dados históricos para comparação é a possibilidade de identificar quais são os pontos a melhorar em cada um dos lotes, quais produzem e quais não produzem os resultados esperados, possibilitando uma melhoria contínua e maior visibilidade do desempenho. 
