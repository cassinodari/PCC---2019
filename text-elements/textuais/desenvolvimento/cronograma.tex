% CRONOGRAMA------------------------------------------------------------------

\chapter{CRONOGRAMA}
\label{chap:cronograma}

\begin{table}[H]
\centering
\begin{tabular}{|l|l|l|l|l|l|l|l|l|l|l|l|} 
\hline
                        & Fev & Mar & Abr & Mai & Jun & Jul & Ago & Set & Out & Nov & Dez  \\ 
\hline
Revisão Bibliográfica   & ~ X & ~ X & ~~  &     &     &     &     &     &     &     &      \\ 
\hline
Especificações          & ~ X & ~ X & ~~  &     &     &     &     &     &     &     &      \\ 
\hline
Estudo das Feramentas   & ~ X & ~ X & ~ X &     &     &     &     &     &     &     &      \\ 
\hline
Implementação           &     &     &     & ~ X & ~ X & ~X  &     &     &     &     &      \\ 
\hline
Testes                  &     &     &     &     &     &     &     & ~X  & ~~  &     &      \\ 
\hline
Análise dos Resultados  &     &     &     &     &     &     &     & ~X  & ~ X &     &      \\ 
\hline
Redação da Monografia   &     &     &     &     & ~ X & ~ X & ~ X & ~X  & ~ X &     &      \\ 
\hline
Entrega da Monografia   &     &     &     &     &     &     &     &     &     & ~ X &      \\ 
\hline
Defesa da Monografia    &     &     &     &     &     &     &     &     &     &     & ~ X  \\ 
\hline
Entrega da Versão Final &     &     &     &     &     &     &     &     &     &     & ~ X  \\
\hline
\end{tabular}
\end{table}

\begin{itemize}
    \item Revisão Bibliográfica – Pesquisa sobre o assunto, análise de trabalhos relacionados e
estudo do conteúdo bibliográfico encontrado.
    \item Especificações – Definição das técnicas utilizadas na implementação e escolha das
ferramentas utilizadas.
    \item Estudo das Ferramentas – Estudo das ferramentas utilizadas na implementação.
    \item Implementação – Implementação do algoritmo de remoção de ruídos nas diferentes
ferramentas.
    \item Testes – Realização de testes de funcionalidade e coleta de dados para análise.
    \item Análise dos Resultados – Analisar os resultados obtidos pelas ferramentas de
programação paralela.
\end{itemize}
