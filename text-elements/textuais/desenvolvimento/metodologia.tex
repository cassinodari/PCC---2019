% METODOLOGIA------------------------------------------------------------------

\chapter{METODOLOGIA}
\label{chap:metodologia}

A metodologia a ser empregada neste trabalho consiste no estudo, especificação, implementação e análise dos resultados obtidos a partir da implementação de uma aplicação para auxiliar no gerenciamento de lotes  de frango de corte, em propriedade ou empresa, utilizando as tecnologias React Native, ReactJS, NodeJS, NestJS e Firebase.

As ferramentas para o desenvolvimento foram escolhidas por serem novas no mercado e por serem muito utilizadas quando se tem um projeto utilizando o mesmo conceito deste. Com isso existe um desafio grande para conhecer as boas práticas que cada tecnologia oferece e também para entender quais são as técnicas que não devem ser utilizadas.

Inicialmente será pesquisado sobre as ferramentas propostas para o desenvolvimento da API, juntamente com as técnicas de utilização do NestJS. Será estudada a documentação destes itens para ter boa prática ao desenvolver as aplicações. Serão pesquisados os casos de uso do produtor que cria aves de corte, para entender as necessidades que se tem ao coletar as informações e, com isso, poder trazer ao produtor um ótimo resultado, facilitando ainda o preenchimento dos formulários no próprio aplicativo que será desenvolvido, fazendo com que não opte mais pela utilização do papel. 

A implementação do servidor da API com será conduzida logo após a coleta de informações junto ao produtor e, mais adiante, será feito o desenvolvimento da aplicação móvel e também da aplicação web.

Como escopo do aplicativo móvel pode-se citar os seguintes itens:

\begin{itemize}
    \item Login;
    \item Sincronização dos dados;
    \item Dados do lote atual;
    \item Gerenciamento da mortalidade;
    \item Registro de pesagem;
    \item Controle da ração;
    \item Acesso ao histórico dos registros;
\end{itemize}
